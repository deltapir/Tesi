\chapter*{Ringraziamenti}
\thispagestyle{empty}
I primi ``Grazie'' vanno ai miei genitori e alle mie sorelle che mi hanno sopportato per 26 anni\footnote{e continueranno a farlo!} e supportato soprattutto in questo ultimo periodo. Vi ringrazio per avermi ``inculcato'' l'amore per la curiosità che mi ha portato ad intraprendere la strada più azzeccata che abbia mai fatto in vita mia: ottenere un pezzo di carta con su scritto che sono ingegnere.

Un saluto particolare al dottore e amico menzionato all'inizio di questa tesi: Giuseppe Sabino.

Ringrazio gli ingegneri che mi hanno formato e permesso di avviare la mia carriera: il \tit{Prof.}{Adolfo Palombo} per aver convogliato quel marasma di idee \emph{green} sull'efficientamento energetico in qualcosa di effettivamente utile;
ringrazio Cesare Forzano e Annamaria Buonomano per la vostra gentilezza innanzitutto e poi per l'aiuto resomi disponibile alla fine\footnote{fine?} di questo percorso;
ringrazio gli ingegneri Cristian Monfrecola e Lucio Pandolfi per avermi sottoposto ad una terapia forzata e necessaria di nonnismo!
Ringrazio l'\tit{Ing.}{Gioacchino Forzano} per la sua umiltà e inspiegabilmente continua disponibilità.\\
Ringrazio tutto il gruppo ``Spuorchi e Scostumati'', ovvero la mia seconda famiglia. O dovrei ringraziare qualcuno ai piani alti per averci fatto incontrare? Sappiate che ognuno di voi è per me una spalla su cui piangere in caso di bisogno, una stanza vuota in cui sfogarmi, un panorama in cui rilassarsi, una compagnia per una passeggiata e una squadra (un po' scarsa!) di calcetto, siete una tavolata imbandita a pasquetta e un pieno di diesel per ogni viaggio, siete un bicchiere di vino per quando bisogna scherzare e un silenzio comprensivo, siete un caldo abbraccio e la corda giusta da pizzicare in un concerto. Vi voglio un mondo di bene e mi ritengo fortunato di avervi conosciuto! Un abbraccio particolare va ad Antonella, Ada, Nicola e Mario perché sì!\\
A Roma e Pisa, invece, saluto il gruppo ``\emph{Ke sfaccimm la Kastla, il KKK, Wagliu e altre creature leggendarie}'' (ovvero i compagni Materazzo, Bucciarelli e Antonelli): siete la più inesauribile fonte di \emph{amicizia}.
\newpage
\thispagestyle{empty}
Ringrazio i miei amici Scout dell'Assoraider che considero da tempi immemori il mio rifugio tranquillo e disordinato dal rumore del mondo: un sentito saluto a Federica\footnote{\textsc{puzzi!}} per la \emph{continua} amicizia, pazienza e perché sì!.\\Ringrazio il gruppo dell'università per la stima/amicizia che ci unisce, per il vostro sempre presente parere scientifico e per le burle: Biagio, Antonio, Gennaro, Pasquale, Giuseppe, Saverio e Luigi.\\
Ringrazio, abbraccio e saluto tutto il resto della famiglia (nonni, zii, cugini, parenti vicini e lontani) perché siamo uniti e in contatto in ogni occasione anche se ci separano chilometri (e in alcuni casi \emph{migliaia} di chilometri).\\
Saluto e ringrazio i miei professori del Liceo Scientifico "B. Pascal" di Pomezia per avermi mostrato, dimostrato e trasmesso nitidamente e a colori la passione per la cultura: \tit{Prof.ssa}{Cadelli}, \tit{Prof.}{Rossi}, \tit{Prof.ssa}{Nardecchia}, \tit{Prof.}{Russo} e la~\tit{Prof.ssa}{Zadra}.

Ognuno di voi mi ha dato qualcosa a proprio modo e per questo vi ringrazio. Spero solo che almeno per un istante da quando vi conosco sia riuscito in qualche modo a ricambiare quanto mi avete dato.