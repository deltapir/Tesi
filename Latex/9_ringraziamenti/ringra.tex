\thispagestyle{empty}
ringraziamenti 1\\
\newpage
\thispagestyle{empty}
ringraziamenti 2
%\begin{flushright}
%\Large\textsf{È fastidioso ringraziare solo in queste occasioni,\\bisognerebbe farlo più spesso.}	
%\end{flushright}
%
%I primi ``Grazie'' vanno ai miei genitori e alle mie sorelle che mi hanno sopportato per 26 anni\footnote{e continueranno a farlo!} e supportato soprattutto in questo ultimo periodo. Vi ringrazio per avermi ``inculcato'' l'amore per la curiosità che mi ha portato ad intraprendere la strada più azzeccata che abbia mai fatto in vita mia: ottenere un pezzo di carta con su scritto che sono ingegnere.
%
%Un saluto particolare al dottore e amico citato all'inizio di questa tesi: Giuseppe Sabino.
%
%Ringrazio gli ingegneri che mi hanno formato e permesso di arrivare in questo punto della mia carriera.\\
%Il \tit{Prof.}{Adolfo Palombo} per aver convogliato quel marasma di idee \emph{green} sull'efficientamento energetico in qualcosa di effettivamente utile;
%ringrazio Cesare Forzano e Annamaria Buonomano per la vostra gentilezza innanzitutto e poi per l'aiuto resomi disponibile alla fine\footnote{fine?} di questo percorso;
%ringrazio gli ingegneri Cristian Monfrecola e Lucio Pandolfi per avermi sottoposto ad una terapia forzata e necessaria di nonnismo!
%Ringrazio l'\tit{Ing.}{Gioacchino Forzano} per la sua umiltà e inspiegabilmente eccessiva disponibilità.\\
%Ringrazio tutto il gruppo ``Spuorchi e Scostumati'', ovvero la mia seconda famiglia. O dovrei ringraziare qualcuno ai piani alti per averci fatto incontrare? Sappiate che ognuno di voi è per me una spalla su cui piangere in caso di bisogno, una stanza vuota in cui sfogarmi, un panorama in cui rilassarsi, una compagnia per una passeggiata e una squadra (un po' scarsa!) di calcetto, siete una tavolata imbandita a pasquetta e un pieno di diesel per ogni viaggio, siete un bicchiere di vino per quando bisogna scherzare e un silenzio comprensivo, siete un caldo abbraccio e la corda giusta da pizzicare in un concerto. Vi voglio un mondo di bene e mi ritengo fortunato di avervi conosciuto! Un abbraccio particolare va ad Antonella, Ada, Nicola e Mario perché sì!\\
%A Roma e Pisa, invece, saluto il gruppo ``\emph{Ke sfaccimm la Kastla, il KKK, Wagliu e altre creature leggendarie}'' (ovvero i compagni Materazzo, Bucciarelli e Antonelli): siete la più inesauribile fonte di \emph{amicizia}. \\
%Ringrazio i miei amici Scout dell'Assoraider che considero da tempi immemori il mio rifugio tranquillo e disordinato dal rumore del mondo: un doveroso saluto va a Federica, Sara e Edoardo.
%\newpage
%\thispagestyle{empty}
%Ringrazio il gruppo dell'università per la stima che ci unisce, per il vostro sempre presente parere scientifico e per le burle: Biagio, Antonio, Gennaro, Pasquale, Giuseppe, Saverio e Luigi.\\
%Ringrazio, abbraccio e saluto tutto il resto della famiglia (nonni, zii, cugini, parenti vicini e lontani) perché siamo uniti e in contatto in ogni occasione anche se ci separano chilometri (e in alcuni casi \emph{migliaia} di chilometri).\\
%Saluto e ringrazio i miei professori del Liceo Scientifico "B. Pascal" di Pomezia per avermi mostrato, dimostrato e trasmesso nitidamente e a colori la passione per la cultura: \tit{Prof.ssa}{Cadelli}, \tit{Prof.}{Rossi}, \tit{Prof.ssa}{Nardecchia}, \tit{Prof.}{Russo} e la~\tit{Prof.ssa}{Zadra}.
%
%Ognuno di voi mi ha dato qualcosa a proprio modo e per questo vi ringrazio. Spero solo che almeno per un istante da quando vi conosco sia riuscito in qualche modo a ricambiare quanto mi avete dato.
%\vspace{2em}
%Lo studio presente in questo elaborato di laurea ha permesso sia di affacciarmi in prima persona al mondo del lavoro (ovvero applicare la teoria appresa all'università), sia di interfacciarmi con la macchina burocratica e con tutto quell'apparato di normative e leggi che, da una parte, consigliano e promuovono la \emph{retta via}, dall'altra, limitano e avviliscono l'operato umano privilegiando la mediocrità senza incoraggiare, invece, l'ingegno e la creatività che contraddistinguono un essere umano dall'altro.

%Consapevole di dovermi un giorno scontrare con poteri e logiche più forti di me, spero e mi
%auguro di avere sempre la forza di seguire le parole di \emph{Baden Powell}, fondatore dello scoutismo: ``\emph{Procurate di lasciare il mondo un po' migliore di come lo avete trovato}''.
%\vspace{1em}

